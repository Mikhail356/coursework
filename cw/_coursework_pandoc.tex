\hypertarget{ux432ux432ux435ux434ux435ux43dux438ux435}{%
\section{Введение}\label{ux432ux432ux435ux434ux435ux43dux438ux435}}

Сейчас Консорциум Всемирной Паутины (W3C) и различные группы людей
сосредоточены также на разработке и поддержке протоколов, структур
данных и спецификаций необходимых для оптимальной работы
децентрализованных интернет-систем (веб-систем) на основе
интернет-технологий (веб-технологий). Они дают возможность объединить
многие существующие ресурсы наукометрических данных в одну
децентрализованную (разделенную на независимых участников) веб-систему.
Вот некоторые из действующих:

\begin{itemize}
\tightlist
\item
  Microformats2 {[}3{]} является набором классов HTML. Создан для того,
  чтобы служить метаданными об элементе, обозначая его как
  представляющий определенный тип данных (например, контактную
  информацию, события, публикации, соавторы).
\item
  Протокол уведомления об упоминаниях Webmention {[}1{]} это
  спецификация W3C, описывающая протокол уведомления любого
  интернет-адреса (URL-адреса), когда другой интернет-сайт (веб-сайт)
  ссылается на него, и для интернет-страниц(веб-страниц) для запроса
  уведомлений, когда кто-то ссылается на них. -- Открытый протокол
  WebSub {[}2{]}. Служит для распределенной связи издателей, подписчиков
  и ресурсов в Интеренете. ресурсе.
\item
  Протокол Solid {[}4{]}, разрабатываемый создателем всемирной паутины
  Сэром Тимоти Джон Бернерсом-Ли.
\end{itemize}

Также существуют статьи по смежным направлениям исследований. Начать
можно с работы
\href{http://128.171.57.22/bitstream/10125/59901/0461.pdf}{Towards a
Decentralized Process for Scientific Publication and Peer Review using
Blockchain and IPFS}. В ней описывается новая система распространения
научных публикаций с использованием IFPS и блокчейна. IFPS это
децентрализованная файловая система, дающая возможность раздавать
информацию в децентрализованной сети одноранговых узлов и использующая
безопасные ссылки. А блокчейн это технология, опирающаяся на книгу
транзакций, которая обновляется и поддерживается сетью одноранговых
узлов. Также он безопасен до тех пор пока хотя бы половина
вычислительной мощности обеспечивается порядочными коллегами. На его
основе развернуты смарт-контракты (как в Ethereum). Технология блокчейна
позволяет запускать на одноранговых узлах небольшие фрагменты кода
(смарт-контракты). Все эти технологии используются в данной работе для
обеспечения системы репутации авторов, комплекса открытого доступа для
рассылки документов и введения прозрачности управления экспертной
оценкой. Первое и последнее предлагается выполнить при помощи
использования блокчейна, а второе используя IFPS. Недостатки данной
работы вижу в отсутствии возможности контроля за распространением
документов, а также необходимость личного заполнения страниц себя и
публикаций. Однако они продиктованы целями авторов, а именно желанием
возобновить движение за свободную науку, так как на данный момент
некоторые издатели по прежнему могут навязывать свою политику и
концентрировать прибыль на себе.

Следующая статья это
\href{http://ceur-ws.org/Vol-2353/paper68.pdf}{Decentralized system for
run services}. В ней авторы предлагают решение вопроса дальнейшего
формирования сетей и их безопасности. А именно вопроса доступности,
повышения скорости получения и использования информации. Это делается
при помощи группы алгоритмов хэширования SHA-2, так как они относительно
безопасны, достаточно исследованы, построены на основе структуры
Меркеля-Демгарда и обладают высокой скоростью выполнения. Также
присутствуют два алгоритма шифрования - ECDH для генерации общего ключа
и AES для дальнейшего шифрования данных, используемых в работе. И еще
TCP алгоритм передачи информации из-за его большей надежности. Целью
работы является внедрение системы поддержки вычислений в
децентрализованных сетях, которая поддерживает работу прикладных
сервисов. Так как в работе предлагается использовать децентрализованную
систему для запуска сервисов над изолированной операционной системой,
это является минусом в некотором роде из-за того что такое решение более
тяжеловесное (а следоваельно может медленнее работать на устройствах с
устаревшими физическими компонентами) по сравнению с клиент-серверным
решением.

Также есть исследование о способах расположения и поиска данных в
одноранговом научном сотрудничестве
\href{https://arxiv.org/pdf/cs/0209031.pdf}{Locating Data in
(Small-World?) Peer-to-Peer Scientific Collaborations}. В нем
рассматривается несколько подходов к хранению и нахождению данных в
малом мире (структуры возникающей в работе
\href{https://epubs.siam.org/doi/pdf/10.1137/S003614450342480?xid=PS_smithsonian}{The
Structure and Function of Complex Networks}). В результате которого
авторы приходят к выводу что научное сотрудничество человека с человеком
может извлечь выгоду из топологии малого мира.

Эти работы представляют явное доказательство того, что научное
сообщество заинтересовано в создании систем распределенного хранения и
вычисления. Однако на данный момент кажется, что вариант перехода сразу
к системе одноранговых сотрудничеств излишен. Так как не все научные
сотрудники обладают нужными компьютерами для поддержания таких систем. В
таком случае выглядит разумным желание частичной децентрализации, где
авторы собираются под разными учреждениями, обеспечивающими поддержание
и функционирование системы. Эта работа рассматривает ее реализацию и
математическую модель на примере частично распределенной
наукометрической системы.

\hypertarget{ux442ux435ux445ux43dux43eux43bux43eux433ux438ux438-ux43fux43eux441ux442ux440ux43eux435ux43dux438ux44f-ux434ux435ux446ux435ux43dux442ux440ux430ux43bux438ux437ux43eux432ux430ux43dux43dux44bux445-web-ux441ux438ux441ux442ux435ux43c--ux43eux431ux437ux43eux440}{%
\section{Технологии построения децентрализованных Web-систем ---
обзор}\label{ux442ux435ux445ux43dux43eux43bux43eux433ux438ux438-ux43fux43eux441ux442ux440ux43eux435ux43dux438ux44f-ux434ux435ux446ux435ux43dux442ux440ux430ux43bux438ux437ux43eux432ux430ux43dux43dux44bux445-web-ux441ux438ux441ux442ux435ux43c--ux43eux431ux437ux43eux440}}

\hypertarget{webmention-1}{%
\subsection{Webmention {[}1{]}.}\label{webmention-1}}

Webmention (веб-упоминание) - это веб-стандарт для упоминаний и
коммуникации в сети. Он является одним из базовых блоков общения для
растущего дерева комментариев, лайков, репостов и других разнообразных
взаимодействий в децентрализованной социальной сети.

\hypertarget{ux43eux431ux449ux438ux439-ux43fux440ux438ux43dux446ux438ux43f-ux440ux430ux431ux43eux442ux44b}{%
\subsubsection{Общий принцип
работы.}\label{ux43eux431ux449ux438ux439-ux43fux440ux438ux43dux446ux438ux43f-ux440ux430ux431ux43eux442ux44b}}

Веб-упоминания отправляются "от" исходного адреса URL ресурса "к"
целевому адресу URL ресурса, чтобы уведомить цель о том, что она была
упомянута в исходном адресе URL ресурса по следующему обобщенному
алгоритму:

\begin{enumerate}
\def\labelenumi{\arabic{enumi}.}
\tightlist
\item
  Пользователь Миша пишет пост в своем блоге.
\item
  Пользователь Женя пишет пост в своем блоге, который ссылается на пост
  Миши.
\item
  После того, как у поста появляется адрес URL, сервер Миши помечает
  упоминание сообщения Жени, как часть процесса публикации.
\item
  Сервер Миши создает упоминание \emph{Webmention} на посту Жени, чтобы
  найти конечную точку своего упоминания \emph{Webmention} (если она не
  найдена, процесс останавливается).
\item
  Сервер Миши отправляет уведомление \emph{Webmention} на конечную точку
  упоминания \emph{Webmention} публикации Жени следующим образом:

  \begin{itemize}
  \tightlist
  \item
    \textbf{источник (source)} назначается на постоянную ссылку
    сообщения Миши
  \item
    \textbf{цель (target)} назначается на постоянную ссылку сообщения
    Жени.
  \end{itemize}
\item
  Сервер Жени получает уведомление \emph{Webmention}.
\item
  Сервер Жени проверяет, что \textbf{target} в \emph{Webmention}
  является допустимой постоянной ссылкой в его блоге (если нет,
  обработка останавливается).
\item
  Сервер Жени проверяет, что источник \textbf{source} в упоминании
  \emph{Webmention} (при извлечении, после FETCH перенаправлений)
  содержит гиперссылку на цель \textbf{target} (если нет, обработка
  останавливается).
\end{enumerate}

\hypertarget{ux43fux440ux43eux442ux43eux43aux43eux43b-websub-2}{%
\subsection{Протокол WebSub
{[}2{]}.}\label{ux43fux440ux43eux442ux43eux43aux43eux43b-websub-2}}

WebSub - предоставляет общий механизм связи между издателями любого вида
веб-контента с их подписчиками, основанный на веб-крюках HTTP-web-hooks.
Запросы на подписку повторно передаются через концентраторы, которые
валидируют и верифицирют запрос (то есть проверяют то что запросы
созданы правильно и то что они созданы такими какими предполагались).
Затем концентраторы распространяют новый или обновленный контент среди
подписчиков, после того как он становится доступен.

\hypertarget{ux43eux441ux43dux43eux432ux43dux44bux435-ux43eux43fux440ux435ux434ux435ux43bux435ux43dux438ux44f}{%
\subsubsection{Основные
определения:}\label{ux43eux441ux43dux43eux432ux43dux44bux435-ux43eux43fux440ux435ux434ux435ux43bux435ux43dux438ux44f}}

\begin{enumerate}
\def\labelenumi{\arabic{enumi}.}
\tightlist
\item
  \textbf{Тема}. Адрес URL ресурса HTTP (HTTPS). Единица, на изменения
  которой можно подписаться.
\item
  \textbf{Концентратор}. Сервер (адрес URL), реализующий обе стороны
  этого протокола. Любой концентратор \emph{МОЖЕТ} реализовать свои
  собственные политики в отношении того кто имеет право им пользоваться.
\item
  \textbf{Издатель}. Владелец темы. Уведомляет концетратор об обновлении
  тематической ленты. Издатель не знает подписчиков, если таковые
  имеются.
\item
  \textbf{Подписчик}. Организация (человек/программа), которая хочет
  получать уведомления об изменениях в теме. Подписчик должен иметь
  прямой доступ к сети и идентифицироваться по адресу URL обратного
  вызова себя.
\item
  \textbf{Подписка}. Уникальное отношение подписчика к теме,
  показывающее что он должен получать обновления этой темы. Уникальным
  ключом подписки является неизменяемый список (кортеж) (адрес URL темы,
  адрес URL обратного вызова подписчика). Подписки могут (если так решит
  концентратор) иметь определенное время действия.
\item
  \textbf{Адрес URL обратного вызова подписчика}. Адрес URL по которому
  подписчик желает получать запросы о распространении контента.
\item
  \textbf{Мероприятие}. Событие, вызывающее обновления нескольких тем
  (возможно одной). Каждое происходящее событие (например «Миша
  опубликовал сообщение в сообществе Linux.») может затронуть несколько
  тем (здесь «Миша опубликовал сообщение.» и «В сообществе Linux
  появилось новое сообщение.»). \emph{Мероприятия} вызывают обновление
  тем. Затем концентратор просматривает все подписки на затронутые темы,
  ища и доставляя контент подписчикам.
\item
  \textbf{Уведомление о распространении контента / (запрос на
  распространение контента)}. Полезная нагрузка, содержащая изменения
  темы или полностью обновленную тему. В зависимости от типа наполнения
  раздела дельта может быть вычислена концентратором и отправлена всем
  подписчикам.
\end{enumerate}

\hypertarget{ux43eux431ux449ux438ux439-ux43fux440ux438ux43dux446ux438ux43f-ux440ux430ux431ux43eux442ux44b-1}{%
\subsubsection{Общий принцип
работы:}\label{ux43eux431ux449ux438ux439-ux43fux440ux438ux43dux446ux438ux43f-ux440ux430ux431ux43eux442ux44b-1}}

\begin{enumerate}
\def\labelenumi{\arabic{enumi}.}
\tightlist
\item
  Подписчики обнаруживают концентратор адреса URL темы и отправляют
  сообщение POST в один или несколько объявленных концентраторов, чтобы
  получать обновления при изменении раздела.
\item
  Издатели уведомляют свои адреса URL концентраторов об изменении
  тем(ы).
\item
  Когда концентратор идентифицирует изменение в разделе, он отправляет
  уведомление о распространении контента всем зарегистрированным
  подписчикам.
\end{enumerate}

\hypertarget{ux44fux437ux44bux43a-ux440ux430ux437ux43cux435ux442ux43aux438-microformats2-3}{%
\subsection{Язык разметки Microformats2
{[}3{]}.}\label{ux44fux437ux44bux43a-ux440ux430ux437ux43cux435ux442ux43aux438-microformats2-3}}

Microformats2-это последняя версия микроформатов, самый простой способ
разметки структурированной информации в HTML. Microformats2 повышает
простоту использования и реализации для авторов (издателей) и
разработчиков (разработчиков парсеров) по сравнению с предыдущей
версией. Главное отличие от других способов разметки файлов HTML состоит
в том, что он поддерживает создание дополнительных классов.

\hypertarget{ux43fux440ux435ux444ux438ux43aux441ux44b-ux434ux43bux44f-ux438ux43cux435ux43d-ux43aux43bux430ux441ux441ux43eux432}{%
\subsubsection{Префиксы для имен
классов.}\label{ux43fux440ux435ux444ux438ux43aux441ux44b-ux434ux43bux44f-ux438ux43cux435ux43d-ux43aux43bux430ux441ux441ux43eux432}}

Все имена классов микроформатов используют префиксы. Префиксы-это
синтаксис, независимый от словарей, которые разрабатываются отдельно.

\begin{enumerate}
\def\labelenumi{\arabic{enumi}.}
\tightlist
\item
  \textbf{h-*} для имен корневых классов (например h-card)
\item
  \textbf{p-*} для свойств обычного текста (например p-name)
\item
  \textbf{u-*} для свойств адреса URL (например u-photo)
\item
  \textbf{dt-*} для свойств даты/времени (например dt-bday)
\item
  \textbf{e-*} для встроенных свойств разметки (например e-note)
\end{enumerate}

\hypertarget{ux440ux430ux441ux43fux440ux435ux434ux435ux43bux435ux43dux43dux430ux44f-ux441ux435ux442ux44c-the-solid-ecosystem-4}{%
\subsection{Распределенная сеть the solid ecosystem
{[}4{]}}\label{ux440ux430ux441ux43fux440ux435ux434ux435ux43bux435ux43dux43dux430ux44f-ux441ux435ux442ux44c-the-solid-ecosystem-4}}

\textbf{Solid} - это промежуточная итерация от изобретателя всемирной
паутины сэра Тима Бернерса-Ли. Она представляет собой оригинальное
видение Тима о Сети как средстве безопасного, децентрализованного обмена
публичными и частными данными.

\hypertarget{ux43eux431ux449ux438ux435-ux43fux43eux43bux43eux436ux435ux43dux438ux44f}{%
\subsubsection{Общие
положения}\label{ux43eux431ux449ux438ux435-ux43fux43eux43bux43eux436ux435ux43dux438ux44f}}

\begin{enumerate}
\def\labelenumi{\arabic{enumi}.}
\tightlist
\item
  На Solid сервере размещается один или несколько Solid модулей,
  доступных по протоколу Solid.
\item
  Модуль, размещенный на Solid сервере, полностью отделен от всех
  остальных. Он имеет свой собственный набор данных и правил доступа и
  полностью контролируется тем, кому он принадлежит (то есть вами).
\item
  Пользователь сам решает, где разместить свой модуль. Клиент может
  выбрать либо разместить модуль самостоятельно, либо модуль размещается
  расширяющейся сетью поставщиков модулей-устройств вместо него.
\item
  Пользователь также может иметь более одного модуля, которые, возможно,
  размещены в разных местах. Это эффективно и прозрачно для приложений и
  служб, которые использует клиент, потому что его данные, где бы они ни
  были размещены, или данные, которыми он поделился, все связаны через
  его личность.
\item
  Пользователь может хранить любые данные в Solid модуле, и может
  определить, кто или что может получить доступ к этим данным на
  детальном уровне, используя системы аутентификации и авторизации
  Solid.
\item
  Хранимые пользователем данные совместимы с другими. Это возможно
  благодаря открытому стандарту форматов. Они могут быть проверены Solid
  сервером, для гарантии сохранения целостности данных после
  взаимодействия с ними разрозненных приложений.
\end{enumerate}

Подводя итог это означает, что вы можете делиться определенными вами
частями ваших данных с другими людьми и группами, которым вы доверяете,
или с экосистемой приложений и сервисов, которые могут читать и
записывать данные в вашем модуле, используя стандартные шаблоны для
взаимодействия приложений. И точно так же, как вы можете поделиться
своими данными с другими, они могут поделиться своими данными с вами.

\hypertarget{ux43cux430ux442ux435ux43cux430ux442ux438ux447ux435ux441ux43aux430ux44f-ux43cux43eux434ux435ux43bux44c-ux434ux435ux446ux435ux43dux442ux440ux430ux43bux438ux437ux43eux432ux430ux43dux43dux43eux439-ux441ux438ux441ux442ux435ux43cux44b}{%
\section{Математическая модель децентрализованной
системы}\label{ux43cux430ux442ux435ux43cux430ux442ux438ux447ux435ux441ux43aux430ux44f-ux43cux43eux434ux435ux43bux44c-ux434ux435ux446ux435ux43dux442ux440ux430ux43bux438ux437ux43eux432ux430ux43dux43dux43eux439-ux441ux438ux441ux442ux435ux43cux44b}}

В прошлом разделе был рассмотрен набор технологий, используемых при
создании децентрализованных систем. Для дальнейших рассуждений и
исследований нужно создать абстрактную модель, описывающую поведение
новой системы в различных ситуациях. В данной работе эта модель будет
упрощенной и в дальнейшем, с ростом проекта, обрастать новыми
ограницениями и объектами в зависимости от потребностей. Для описания
основных объектов и взаимодействий используется язык теории графов.
Появление новых понятий необходимо для более четкого представления
модели.

\hypertarget{ux43eux43fux440ux435ux434ux435ux43bux435ux43dux438ux44f}{%
\subsection{Определения}\label{ux43eux43fux440ux435ux434ux435ux43bux435ux43dux438ux44f}}

Пучок - связный неориентированный граф со структурой зависящей от
времени.

Центр (пучка) - одна из вершин пучка, через которую могут проходить
потоки сообщений (может быть несколько у каждого пучка).
Идентифицируется путем от корня для владеющего ими пучка.

Сообщение - функция от времени, источника и цели, которая действует на
множестве центров и путей их соединяющих.

Источник (сообщения) - один из листьев отправляющего сообщение центра.

Цель (сообщения) - один из листьев получающего сообщение центра.

Поток (сообщений) - последовательный набор сообщений.

Корень (пучка) - центр пучка, который принимает потоки сообщений от
других пучков и отправляет им ответные потоки сообщений. Имеет
уникальный идентификатор, определяемый специальной функцией от времени и
количества уже появлявшихся в системе пучков с областью значений в не
более чем счетном множестве имен.

Лист (пучка) - не являющаяся центром вершина пучка. Может служить только
целью или источником сообщений (или страница ученого без публикаций).
Идентифицируется путем от корня для владеющего ими пучка.

Путь - последовательность ребер, где 2 соседних ребра объединены общей
вершиной.

Система - функция от времени и существующих пучков, где область значений
это граф отображающий пучки точками, а существующие в данный момент пути
между ними кривыми.

Объекты взаимодействуют друг с другом путем отправления сообщений по
путям. Они могут отправляться и приниматься только серверами по путям
определенным системой и внутренней структурой пучков. Цель и источник
узнаются используя дополнительные сообщения.

Имеет смысл ввести некоторые дополнения к текущему набору определений
для придания им прикладного смысла. Начать давать описание следует в
порядке введения объектов. Пучок это научное учреждение или любая другая
организация, обладающая центром (сервером для приема и отправки
сообщений) и листьями (веб-страница публикации, сотрудника или
организации). Система есть веб-система, объединяющая все существующиие
пучки, даже те которые в данный момент к ней не подсоединены, но были
таковыми ранее. Она не позволяет напрямую менять одному центру листья
другого. Сообщение же является ничем иным как HTTP запросом или набором
HTTP запросов со специальными полями источник, цель. Ребро это
какая-либо связь между центрами, листьями и друг с другом (провода по
которым проходит сигнал, беспроводные адаптеры).

В такой постановке пусть у нас есть текущее состояние системы \$S(t,
P)\$, где \$t\$ - дискретное время, а \$P = (p\_1(t), p\_2(t), ...,
p\_n(t))\$ набор пучков подключенных к системе в данный момент.
Рассмотрим пример.

\hypertarget{ux440ux435ux430ux43bux438ux437ux430ux446ux438ux44f-ux43cux430ux43aux435ux442ux430-ux434ux435ux446ux435ux43dux442ux440ux430ux43bux438ux437ux43eux432ux430ux43dux43dux43eux439-ux441ux438ux441ux442ux435ux43cux44b}{%
\section{Реализация макета децентрализованной
системы}\label{ux440ux435ux430ux43bux438ux437ux430ux446ux438ux44f-ux43cux430ux43aux435ux442ux430-ux434ux435ux446ux435ux43dux442ux440ux430ux43bux438ux437ux43eux432ux430ux43dux43dux43eux439-ux441ux438ux441ux442ux435ux43cux44b}}

В данном разделе приводится подробное описание работы системы и набора
используемых технологий. Это нужно для создания физического объекта, на
котором будут проверяться математические выводы получнные на модели
предыдущего раздела. Также бывает полезным для построения таких выводов
явно представлять себе в голове всю структуру системы. Перейдем к
описанию основных рабочих объектов.

\hypertarget{ux43eux43fux440ux435ux434ux435ux43bux435ux43dux438ux44f-1}{%
\subsection{Определения}\label{ux43eux43fux440ux435ux434ux435ux43bux435ux43dux438ux44f-1}}

\begin{enumerate}
\def\labelenumi{\arabic{enumi}.}
\tightlist
\item
  Страница ученого - html страница содержащая информацию о ученом, его
  соавторах и публикациях.
\item
  Страница публикации - html страница содержащая информацию о публикации
  и ее авторах.
\item
  Сервер - сервер, которым владеет определенное научное учреждение. На
  нем содержатся страницы ученых и их публикаций.
\item
  Источник (source) - страница ученого/публикации с которой отправляют
  web интернет запросы.
\item
  Цель (target) - страница ученого/публикации на которую поступают web
  интернет запросы.
\item
  WebMention сервер - специальный сервер, который получает и записывает
  в базу данных запросы от серверов\_ в определенном формате: URL адрес
  source источника, URL адрес target цели, URL адрес WebMention сервера.
  Отличается своей структурой от сервера, но также является
  собственностью соответствующего серверу научного учреждения.
\item
  База Данных (БД) - хранит записи в формате: URL адрес source
  источника, URL адрес target цели, время поступления запроса на
  WebMention сервер, URL адрес WebMention сервера. Является
  собственностью соответствующего серверу и WebMention серверу научного
  учреждения.
\end{enumerate}

\hypertarget{ux43aux43eux43cux430ux43dux434ux44b-ux440ux430ux431ux43eux442ux44b-ux441-ux43fux440ux43eux435ux43aux442ux43eux43c}{%
\subsection{Команды работы с
проектом}\label{ux43aux43eux43cux430ux43dux434ux44b-ux440ux430ux431ux43eux442ux44b-ux441-ux43fux440ux43eux435ux43aux442ux43eux43c}}

Чтобы дать физическое оформление способам связи из предыдущего раздела в
проекте используется язык программирования python3. Здесь кратко
описывается поведение этого кода. В работе проекта используется всего
две команды это команда обновления и проверки информации в базе данных
веб-запросов пользователя. Такой малый набор команд связан с его
логической простотой и легкостью в использовании.

\hypertarget{ux43aux43eux43cux430ux43dux434ux430-ux43eux431ux43dux43eux432ux43bux435ux43dux438ux44f-update}{%
\subsubsection{Команда обновления
(update)}\label{ux43aux43eux43cux430ux43dux434ux430-ux43eux431ux43dux43eux432ux43bux435ux43dux438ux44f-update}}

Начать стоит с описания поведения команды обновления. На вход ей
подается URL адрес страницы. Команда отпраляет все ссылки в ней на
WebMention сервер, который записывает их в соответствующем формате в БД.
Используется перед удалением страницы или после изменения содержимого
страницы, не затрагивающего ссылки. Дадим более детальное с технической
точки зрения описание. Алгоритм работы:

\begin{enumerate}
\def\labelenumi{\arabic{enumi}.}
\tightlist
\item
  Отправить GET запрос на указанный URL адрес
\item
  Если ответ пришел без ошибок (с кодом 200), то перейти к следующему
  шагу. Иначе прекратить работу и вывести сообщение об ошибке.
\item
  Разобрать html код страницы, полученный на прошлом шаге.
\item
  Найти все ссылки на другие страницы и URL адрес WebMention сервера.
\item
  Отправить на URL адрес WebMention сервера POST запросы, содержащие:

  \begin{enumerate}
  \def\labelenumii{\arabic{enumii}.}
  \tightlist
  \item
    URL Адрес личного WebMention сервера для сервера, содержащего
    текущую страницу.
  \item
    URL Адрес текущей страницы.
  \item
    URL Адрес целевой страницы.
  \end{enumerate}
\item
  Вывод URL адресов обновленных страниц на экран.
\end{enumerate}

\hypertarget{ux43aux43eux43cux430ux43dux434ux430-ux43fux440ux43eux432ux435ux440ux43aux438-check}{%
\subsubsection{Команда проверки
(check)}\label{ux43aux43eux43cux430ux43dux434ux430-ux43fux440ux43eux432ux435ux440ux43aux438-check}}

Приступим к формальному описанию следующей команды, а именно проверки.
На вход ей подается физический адрес .html файла страницы и ее URL
адрес. Команда скачивает из БД свои запросы, появившиеся после последней
проверки, и редактирует существующую страницу в соответствии с ними.
Редактирование происходит путем добавления и удаления новых ученых,
публикаций, ссылок. Используется после изменения ссылок внутри страницы.
Перейдем к формальному описанию выполнения команды. Алгоритм работы:

\begin{enumerate}
\def\labelenumi{\arabic{enumi}.}
\tightlist
\item
  Считать из отдельного файла (с именем time) время последнего
  обновления страницы.
\item
  Разобрать .html страницу на теги.
\item
  Найти в разборе URL адрес WebMention сервера.
\item
  Считать из БД все записи, посланные на данный WebMention сервер,
  имеющие целью текущую страницу, и появившиеся после даты последней
  проверки.
\item
  Отправить GET запросы на все WebMention сервера с целью на страницы,
  запросы которых содержали целью текущую страницу.
\item
  Если результат запроса оказался удовлетворительным (вернулся с кодом
  200) перейти к следующему пункту, иначе поместить обрабатываемую
  страницу в список адресов страниц на удаление (с именем remove).
\item
  Провести проверку страницы на подлинность, в случае успеха перейти к
  следующему шагу, иначе поместить данную страницу в список адресов
  страниц на удаление (с именем remove).
\item
  URL адрес страницы поместить в массив адресов страниц для добавления
  (с именем add).
\item
  Найти в разборе текущей страницы все ссылки на другие страницы и
  поместить в массив current.
\item
  Если среди последних есть адреса страниц подлежащих удалению
  (находятся в remove), то сделать соответствующие виртуальные изменения
  в коде страницы.
\item
  Если в current нет каких-то URL адресов из массива add, то сделать
  соответствующие виртуальные изменения в коде страницы.
\item
  Вывести на экран список изменений страницы и запросить пользователя
  разрешение на их внесение в физический код страницы.
\item
  В случае отказа, завершить работу команды. Иначе провести изменения
  кода страницы.
\item
  Записать в файл time время последнего обновления данной страницы.
\item
  Вызвать команду update, описанную выше.
\end{enumerate}

\hypertarget{ux440ux430ux431ux43eux442ux430-ux441-microformats2}{%
\subsection{Работа с
microformats2}\label{ux440ux430ux431ux43eux442ux430-ux441-microformats2}}

Для работы вышеописанных алгоритмов нужно описать как именно строится
страница и какие используются имена классов в microformats2 формате.
Пример разметки страницы с ее использованием дает общее рпедставление об
этом.

\begin{Shaded}
\begin{Highlighting}[]
\DataTypeTok{\textless{}!DOCTYPE }\NormalTok{HTML}\DataTypeTok{\textgreater{}}
\KeywordTok{\textless{}html\textgreater{}}
  \KeywordTok{\textless{}head\textgreater{}}
   \KeywordTok{\textless{}meta}\OtherTok{ http{-}equiv=}\StringTok{"content{-}type"}\OtherTok{ content=}\StringTok{"text/html"}\OtherTok{ charset=}\StringTok{"utf{-}8"}\KeywordTok{\textgreater{}}
   \KeywordTok{\textless{}title\textgreater{}}\NormalTok{Кривчиков Максим Александрович}\KeywordTok{\textless{}/title\textgreater{}}
   \KeywordTok{\textless{}link}\OtherTok{ rel=}\StringTok{"webmention"}\OtherTok{ href=}\StringTok{"http://127.0.0.1:1234"} \KeywordTok{/\textgreater{}}
  \KeywordTok{\textless{}/head\textgreater{}}
  \KeywordTok{\textless{}body\textgreater{}}
    \KeywordTok{\textless{}div}\OtherTok{ class=}\StringTok{"h{-}card"}\KeywordTok{\textgreater{}}
      \KeywordTok{\textless{}a}\OtherTok{ class=}\StringTok{"p{-}name"}\KeywordTok{\textgreater{}}\NormalTok{Кривчиков Максим Александрович}\KeywordTok{\textless{}/a\textgreater{}}
      \KeywordTok{\textless{}p\textgreater{}}
      \KeywordTok{\textless{}a}\OtherTok{ class=}\StringTok{"p{-}honorific{-}suffix"}\KeywordTok{\textgreater{}}\NormalTok{Кандидат ф.{-}м. наук}\KeywordTok{\textless{}/a\textgreater{}}\NormalTok{,}
      \KeywordTok{\textless{}a}\OtherTok{ class=}\StringTok{"p{-}role"}\KeywordTok{\textgreater{}}\NormalTok{доцент кафедры вычислительной математики}\KeywordTok{\textless{}/a\textgreater{}}\NormalTok{.}
      \KeywordTok{\textless{}/p\textgreater{}}
    \KeywordTok{\textless{}/div\textgreater{}}
   \KeywordTok{\textless{}/body\textgreater{}}
\KeywordTok{\textless{}/html\textgreater{}}
\end{Highlighting}
\end{Shaded}

Ее разбор в json файле будет выглядеть так

\begin{Shaded}
\begin{Highlighting}[]
\FunctionTok{\{}
    \DataTypeTok{"items"}\FunctionTok{:} \OtherTok{[}
        \FunctionTok{\{}
            \DataTypeTok{"type"}\FunctionTok{:} \OtherTok{[}
                \StringTok{"h{-}card"}
            \OtherTok{]}\FunctionTok{,}
            \DataTypeTok{"properties"}\FunctionTok{:} \FunctionTok{\{}
                \DataTypeTok{"name"}\FunctionTok{:} \OtherTok{[}
                    \StringTok{"Кривчиков Максим Александрович"}
                \OtherTok{]}\FunctionTok{,}
                \DataTypeTok{"honorific{-}suffix"}\FunctionTok{:} \OtherTok{[}
                    \StringTok{"Кандидат ф.{-}м. наук"}
                \OtherTok{]}\FunctionTok{,}
                \DataTypeTok{"role"}\FunctionTok{:} \OtherTok{[}
                    \StringTok{"доцент кафедры вычислительной математики"}
                \OtherTok{]}
            \FunctionTok{\}}
        \FunctionTok{\}}
    \OtherTok{]}\FunctionTok{,}
    \DataTypeTok{"rels"}\FunctionTok{:} \FunctionTok{\{}
        \DataTypeTok{"webmention"}\FunctionTok{:} \OtherTok{[}
            \StringTok{"http://127.0.0.1:1234"}
        \OtherTok{]}
    \FunctionTok{\},}
    \DataTypeTok{"rel{-}urls"}\FunctionTok{:} \FunctionTok{\{}
        \DataTypeTok{"http://127.0.0.1:1234"}\FunctionTok{:} \FunctionTok{\{}
            \DataTypeTok{"text"}\FunctionTok{:} \StringTok{""}\FunctionTok{,}
            \DataTypeTok{"rels"}\FunctionTok{:} \OtherTok{[}
                \StringTok{"webmention"}
            \OtherTok{]}
        \FunctionTok{\}}
    \FunctionTok{\},}
    \DataTypeTok{"debug"}\FunctionTok{:} \FunctionTok{\{}
        \DataTypeTok{"description"}\FunctionTok{:} \StringTok{"mf2py {-} microformats2 parser for python"}\FunctionTok{,}
        \DataTypeTok{"source"}\FunctionTok{:} \StringTok{"https://github.com/microformats/mf2py"}\FunctionTok{,}
        \DataTypeTok{"version"}\FunctionTok{:} \StringTok{"1.1.2"}\FunctionTok{,}
        \DataTypeTok{"markup parser"}\FunctionTok{:} \StringTok{"html5lib"}
    \FunctionTok{\}}
\FunctionTok{\}}
\end{Highlighting}
\end{Shaded}

Видно что использован специальный микроформат h-card и его аргументы
p-name, p-honorific-suffux, p-role для разметки информации.
Примечательно, что все дочерние теги могут иметь любые имена после
определенной приставки, как и описано в обзоре технологий построения
распределенных веб-систем. Также в более сложных страницах для описания
соавторов, публикаций и авторов публикаций используются имена
h-istina-coauthors, h-istina-article-journal, h-istina-authors
соответственно.

\hypertarget{ux437ux430ux43fux443ux441ux43a-ux43cux430ux43aux435ux442ux430}{%
\subsection{Запуск
макета}\label{ux437ux430ux43fux443ux441ux43a-ux43cux430ux43aux435ux442ux430}}

В подтверждение реализации всех описанных выше алгоритмов и
использования описанных технологий представляется разумным дать описание
запуска и работы с уже существующим проектом. Найти его можно в GitHub
репозитории {[}5{]}. Для его запуска нужно выполнить несколько несложных
команд:

\begin{enumerate}
\def\labelenumi{\arabic{enumi}.}
\tightlist
\item
  Скачать и установить все необходимые python библиотеки : mf2py,
  mf2util, json, sys, requests, aiosqlite, sqlite3, bs4, а также версию
  python не ниже 3.
\item
  Запустить команду start.sh (создает сервера при помощи darkhttpd),
  находится в папке serv.
\item
  Запустить при помощи команды python3 webmention-logger.py WebMention
  сервер.
\end{enumerate}

После этих команд запустится несколько локальных серверов которые будут
раздавать документы из соответствующих им папок. Так, например,
содержимое папки serv/1 будет доступно по пути
\url{http://127.0.0.1:8081/} с добавлением соответствующего названия
файла.

На момент написания данной работы макет обрабатывает только 2 ситуации:

\begin{enumerate}
\def\labelenumi{\arabic{enumi}.}
\tightlist
\item
  Пользователь решил обновить ссылки определенной страницы (например
  serv/1/index.html) в базе данных, тогда нужно ввести команду python3
  update.py \url{http://127.0.0.1:8081/index.html}.
\item
  Пользователь хочет проверить действительны ли еще ссылки текущей
  страницы (serv/1/index.html) и нужно ли их дополнить новыми. Тогда
  нужно ввести команду python3 check.py serv/1/index.html
  \url{http://127.0.0.1:8081/index.html}. В ходе работы нужно будет
  подтвердить или отвергнуть изменения текущей страницы.
\end{enumerate}

\hypertarget{ux437ux430ux43aux43bux44eux447ux435ux43dux438ux435}{%
\section{Заключение}\label{ux437ux430ux43aux43bux44eux447ux435ux43dux438ux435}}

кратко описать, что получилось в итоге, сформулировать, что хочется
делать дальше, и желательно добавить планы на далёкое будущее и
интересные проблемы --- некоторые мы с вами обсуждали, другие, возможно,
придумаются в процессе.

\hypertarget{ux441ux43fux438ux441ux43eux43a-ux43bux438ux442ux435ux440ux430ux442ux443ux440ux44b}{%
\section{Список
литературы}\label{ux441ux43fux438ux441ux43eux43a-ux43bux438ux442ux435ux440ux430ux442ux443ux440ux44b}}

\begin{center}\rule{0.5\linewidth}{0.5pt}\end{center}

вообще, его можно делать с помощью pandoc-citeproc, но если не получится
на google scholar найти ещё штук 5 интересных статей для добавления в
обзор, вам проще будет имеющиеся 4 ссылки оформить по ГОСТ вручную

\begin{center}\rule{0.5\linewidth}{0.5pt}\end{center}

\begin{enumerate}
\def\labelenumi{\arabic{enumi}.}
\tightlist
\item
  \url{https://www.w3.org/TR/webmention/}
\item
  \url{https://www.w3.org/TR/websub/}
\item
  \url{http://microformats.org/wiki/microformats2}
\item
  \url{https://solidproject.org/}
\item
  \url{https://github.com/Mikhail356/coursework}
\end{enumerate}
